\documentclass[11pt]{article}

\usepackage[left=3cm,right=3cm,top=3cm,bottom=3cm]{geometry}
\usepackage{amssymb,amsmath,amsthm}
\usepackage{mathtools}
\usepackage{enumitem}
\usepackage{tabularx}
\usepackage{tikz-cd}
\usepackage{parskip}

\newcolumntype{Y}{>{\centering\arraybackslash}X}

\DeclarePairedDelimiter\trunc\lVert\rVert

\newcommand{\U}{\mathcal{U}}
\newcommand{\N}{\mathbb{N}}
\newcommand{\Z}{\mathbb{Z}}
\newcommand{\F}{\mathbb{F}}
\newcommand{\id}{\mathsf{id}}
\newcommand{\ap}{\mathsf{ap}}

\newcommand{\Fin}{\mathsf{Fin}}
\newcommand{\refl}{\mathsf{refl}}
\newcommand{\base}{\mathsf{base}}
\newcommand{\Sloop}{\mathsf{loop}}
\newcommand{\indeq}{\mathsf{ind\mbox{-}eq}}
\newcommand{\inv}{\mathsf{inv}}

\newcommand{\Nsucc}{S}
\newcommand{\Nadd}{\mathsf{add}}
\newcommand{\Nmul}{\mathsf{mul}}

\newcommand{\fst}{\mathsf{pr}_1}
\newcommand{\snd}{\mathsf{pr}_2}

\newcommand{\isEquiv}{\mathsf{is\mbox{-}equiv}}
\newcommand{\isConst}{\mathsf{is\mbox{-}constant}}

\newcommand{\fix}{\mathsf{fix}}


\begin{document}

\title{Introduction to homotopy type theory: exam}
\author{DAT235/DIT577/PhD reading course}
\date{2024, January 12}

\maketitle

\begin{itemize}
\item
Grade scale:\qquad
\begin{tabularx}{10cm}{|c|Y|Y|Y|Y|}
  \hline
  Fraction of points & $\geq 0$ & $\geq 2/5$ & $\geq 3/5$ & $\geq 4/5$
  \\\hline
  Grade & U & 3 & 4 & 5
  \\\hline
\end{tabularx}
\item
Time: 4 hours
\item
No aids allowed.
\item
You may use familiar facts from the course book or our discussions without justification, provided they do not already include the statement to be proven or depend on it.
\item
The axioms of function extensionality and univalence may only be used where stated. 
\end{itemize}

\newpage

\begin{enumerate}

\item \label{path-symmetry}
\textbf{[4 points]}
Fix a type $A$.
The equality type of $A$ has an induction principle involving
\[
\indeq_{a,P} : P(a, \refl_a) \to \prod_{x : A} \prod_{p : a = x} P(x, p)
\]
for $a : A$ and a family of types $P(x, p)$ indexed by $x : A$ and $p : a = x$.

Construct the following elements.
Explicitly state the parameter $P$ when you use $\indeq$.
\begin{enumerate}
\item \label{path-symmetry:sym}
$\displaystyle{
\mathsf{f} : \prod_{x, y : A} x = y \to y = x
},$
\item \label{path-symmetry:sym-coh}
$\displaystyle{
\mathsf{g} : \prod_{x, y : A} \prod_{p : x = y} f(y, x, f(x, y, p)) = p
}.$
\end{enumerate}

\item \label{truncation-level}
\textbf{[4 points]}
Given a type $A$, define what it means:
\begin{enumerate}
\item \label{truncation-level:contractible}
for $A$ to be contractible,
\item \label{truncation-level:general}
for $A$ to have truncation level $n$ where $n : \Z_{\geq -2}$.
\end{enumerate}

\item \label{rolling-rule}
\textbf{[4 points]}
The type of \emph{fixpoints} of a function $u : X \to X$ is defined as
\[
\fix(u) := \sum_{x:X} u(x) = x
\rlap{.}
\]
Given $f : A \to B$ and $g : B \to A$, show that $\fix(g \circ f) \simeq \fix(f \circ g)$.

\item \label{equality-proposition}
\textbf{[4 points]}
Given a type $A$, show that the following are logically equivalent for $x, y : A$:
\begin{enumerate}[label={(\arabic*)}]
\item \label{equality-proposition:truncation}
the propositional truncation $\trunc{x = y}$,
\item \label{equality-proposition:impredicative}
$Q(x) \simeq Q(y)$ for all families $Q$ of propositions indexed by $A$.
\end{enumerate}

\item \label{grothendieck-correspondence}
\textbf{[4 points]}
Let $\U$ be a univalent universe with $I : \U$.
We have a function
\[
h : \U^I \to \sum_{X : \U} I^X
\]
sending $Y : I \to U$ to $(\sum_{i:I} Y(i), \fst)$.
Define a function $k$ in the opposite direction with $k \circ h \sim \id$.
You may use function extensionality.

\item \label{circle-idempotent}
\textbf{[4 points]}
Consider $f : S^1 \to S^1$ with $H : f \circ f \sim f$.
Show that
\[
\isEquiv(f) + \trunc{\isConst(f)}
\rlap{.}
\]
You may use function extensionality and univalence.
\end{enumerate}

\end{document}
